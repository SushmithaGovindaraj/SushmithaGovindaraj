%------------------------------------------------------------------------------
% Lebenslauf in Latex
% Basierend auf: https://github.com/sb2nov/resume und Jake's Resume on Overleaf
% Angepasst für: Sushmitha Govindaraj
% Lizenz : MIT
%------------------------------------------------------------------------------

\documentclass[A4,11pt]{article}
\usepackage{latexsym}
\usepackage[empty]{fullpage}
\usepackage{titlesec}
\usepackage{marvosym}
\usepackage[usenames,dvipsnames]{color}
\usepackage{verbatim}
\usepackage{enumitem}
\usepackage[hidelinks]{hyperref}
\usepackage[german]{babel}
\usepackage{tabularx}
\usepackage{tikz}
\input{glyphtounicode}

%-----SCHRIFTART---------------------------------------------------------------
\usepackage{palatino}

%-----SEITENEINRICHTUNG--------------------------------------------------------

% Seitenränder anpassen
\addtolength{\oddsidemargin}{-1cm}
\addtolength{\evensidemargin}{-1cm}
\addtolength{\textwidth}{2cm}
\addtolength{\topmargin}{-1cm}
\addtolength{\textheight}{2cm}

\urlstyle{same}

\raggedbottom
\raggedright
\setlength{\tabcolsep}{0cm}

% Abschnittsformatierung
\titleformat{\section}{
  \vspace{-4pt}\scshape\raggedright\large
}{}{0em}{}[\color{black}\titlerule \vspace{-5pt}]

% Sicherstellen, dass .pdf maschinenlesbar/ATS-analysierbar ist
\pdfgentounicode=1

%-----BENUTZERDEFINIERTE BEFEHLE FÜR FORMATIERUNGSABSCHNITTE------------------
\newcommand{\CVItem}[1]{
  \item\small{
    {#1 \vspace{-2pt}}
  }
}

\newcommand{\CVSubheading}[4]{
  \vspace{-2pt}\item
    \begin{tabular*}{0.97\textwidth}[t]{l@{\extracolsep{\fill}}r}
      \textbf{#1} & #2 \\
      \small#3 & \small #4 \\
    \end{tabular*}\vspace{-7pt}
}

\newcommand{\CVSubSubheading}[2]{
    \item
    \begin{tabular*}{0.97\textwidth}{l@{\extracolsep{\fill}}r}
      \text{\small#1} & \text{\small #2} \\
    \end{tabular*}\vspace{-7pt}
}

\newcommand{\CVSubItem}[1]{\CVItem{#1}\vspace{-4pt}}

\renewcommand\labelitemii{$\vcenter{\hbox{\tiny$\bullet$}}$}

\newcommand{\CVSubHeadingListStart}{\begin{itemize}[leftmargin=0.5cm, label={}]}
\newcommand{\CVSubHeadingListEnd}{\end{itemize}}
\newcommand{\CVItemListStart}{\begin{itemize}}
\newcommand{\CVItemListEnd}{\end{itemize}\vspace{-5pt}}

%------------------------------------------------------------------------------
% LEBENSLAUF BEGINNT HIER  %
%------------------------------------------------------------------------------
\begin{document}

%-----KOPFZEILE----------------------------------------------------------------

\begin{minipage}[c]{0.05\textwidth}
\-\
\end{minipage}
\begin{minipage}[c]{0.2\textwidth}
\begin{tikzpicture}
    \clip (0,0) circle (1.75cm);
    \node at (0,-.7) {\includegraphics[width = 6.5cm]{IMG_3265.jpg}}; 
    % Falls erforderlich, kann das Bild durch Ändern der at-Koordinaten verschoben werden
    % width definiert den 'Zoom' des Bildes
\end{tikzpicture}
\hfill\vline\hfill
\end{minipage}
\begin{minipage}[c]{0.6\textwidth}
    \textbf{\Huge \scshape{Sushmitha Govindaraj}} \\ \vspace{1pt} 
    \small{+49 1635211451} \\
    \small{Dortmund, Deutschland} \\
    \href{mailto:sushmitharaj2000@gmail.com}{\underline{sushmitharaj2000@gmail.com}}\\
    \href{https://linkedin.com/in/sushmitha-govindaraj-40534a196}{\underline{linkedin.com/in/sushmitha-govindaraj}}
\end{minipage}

%-----ZUSAMMENFASSUNG----------------------------------------------------------
\section{Zusammenfassung}
Robotikingenieurin mit Expertise in KI-gestützten autonomen Systemen, Sensordatenfusion und industrieller Automatisierung. Schwerpunkt auf Deep Learning, Reinforcement Learning, SLAM und robotischer Manipulation. Erfahrung in der Entwicklung intelligenter Systeme zur Verbesserung von Autonomie, Anpassungsfähigkeit und Effizienz in der Robotik durch fortschrittliche KI-Technologien.

%-----KOMPETENZEN--------------------------------------------------------------
\section{Kompetenzen}
 \begin{itemize}[leftmargin=0.5cm, label={}]
    \small{\item{
     \textbf{KI \& Maschinelles Lernen}{: Deep Learning, Reinforcement Learning, Computer Vision, PyTorch, TensorFlow, LSTM} \\
     \textbf{Robotik \& Autonome Systeme}{: ROS1/ROS2, SLAM, Sensorfusion, Navigation, Gazebo, MoveIt} \\
     \textbf{Programmierung}{: Python, C++, MATLAB} \\
     \textbf{Automatisierung}{: SPS-Programmierung, SCADA, Roboterintegration, OpenCV} \\
     \textbf{Embedded Systems}{: SI/PI-Design, SPICE-Simulation, Mikrostreifendesign} \\
     \textbf{Sprachen}{: Englisch (C2), Deutsch (B2 - Lernend), Tamil (Muttersprache)} \\
    }}
 \end{itemize}

%-----AUSBILDUNG---------------------------------------------------------------
\section{Ausbildung}
  \CVSubHeadingListStart
    \CVSubheading
      {{Master of Science $|$ \emph{\small{Automatisierung \& Robotik}}}}{Okt. 2022 -- Gegenwart}
      {TU Dortmund}{Dortmund, Deutschland}
    \CVSubheading
      {{Bachelor of Engineering $|$ \emph{\small{Robotik \& Automatisierung}}}}{2018 -- 2022}
      {PSG College of Technology}{Coimbatore, Indien}
  \CVSubHeadingListEnd

%-----BERUFSERFAHRUNG----------------------------------------------------------
\section{Berufserfahrung}
  \CVSubHeadingListStart
    \CVSubheading
      {Masterarbeitsstudentin}{Nov. 2024 -- Juni 2025}
      {TU Dortmund}{Dortmund, Deutschland}
      \CVItemListStart
        \CVItem{Forschung zu sonarbasiertem SLAM für GNSS-freie Lokalisierung autonomer Schiffe in Unterwasserumgebungen}
        \CVItem{Entwicklung eines Lokalisierungssystems mit IMU-, DVL- und Sonarsensoren in ROS2-Framework}
        \CVItem{Entwurf eines LSTM-neuronalen Netzwerkmodells zur Verbesserung der Odometrie-Genauigkeit unter Wasser}
        \CVItem{Implementierung von Extended Kalman Filter Algorithmen für Echtzeit-Positionierung in autonomer Navigation}
      \CVItemListEnd
    \CVSubheading
      {Wissenschaftliche Hilfskraft}{Juni 2024 -- Mai 2025}
      {TU Dortmund - Labor für Informationsverarbeitung}{Dortmund, Deutschland}
      \CVItemListStart
        \CVItem{Optimierung der Signalverarbeitung in eingebetteten Systemen durch KI-gesteuerte Modellierung}
        \CVItem{Anwendung maschineller Lerntechniken zur Verbesserung der Schaltungsleistung und Signalintegritätsanalyse}
        \CVItem{Entwicklung von Python-Tools für Modelltraining, Datenvorverarbeitung und automatisierte Evaluierungsframeworks}
      \CVItemListEnd
    \CVSubheading
      {Projektpraktikantin}{Dez. 2021 -- Mai 2022}
      {AAtek Robo Private Limited}{Coimbatore, Indien}
      \CVItemListStart
        \CVItem{Entwicklung eines robotischen Automatisierungssystems für RTPCR-Probenentnahme und -verarbeitung}
        \CVItem{Implementierung von Computer-Vision-Algorithmen mit Python und ROS-Framework für präzise Robotersteuerung}
        \CVItem{Verbesserung der Laborautomatisierung durch intelligente Steuerungssysteme und adaptive Mechanismen}
      \CVItemListEnd
  \CVSubHeadingListEnd

%-----PROJEKTE UND FORSCHUNG---------------------------------------------------
\section{Projekte und Forschung}
  \CVSubHeadingListStart
    \CVSubheading
      {{KI-basiertes SI/PI-konformes PCB-Design} $|$ \emph{\small{Python, SPICE, Zuken eCADSTAR}}}{Okt. 2023 -- Juni 2024}
      {TU Dortmund}{}
      \CVItemListStart
        \CVItem{Entwicklung KI-basierter Modelle zur Optimierung von Hochgeschwindigkeits-PCB-Designs für Signalintegrität}
        \CVItem{Erstellung umfassender Trainingsdatensätze mit SPICE-Simulationen und Zuken eCADSTAR}
        \CVItem{Integration KI-gesteuerter Empfehlungen zur Verbesserung der PCB-Design-Zuverlässigkeit und -Leistung}
      \CVItemListEnd
    \CVSubheading
      {{Roboterarm auf omnidirektionaler Basis} $|$ \emph{\small{ROS, Python, C++}}}{Dez. 2021 -- Mai 2022}
      {PSG College of Technology}{}
      \CVItemListStart
        \CVItem{Entwicklung eines 5-achsigen Roboterarms auf Mecanum-Rad omnidirektionaler Basis mit präziser Steuerung}
        \vspace{-12pt}
        \CVItem{Implementierung von Steuerungsalgorithmen für Pick-and-Place-Operationen in dynamischen Umgebungen}
        \CVItem{Integration flexibler Mobilitätslösungen für adaptive Handhabung in verschiedenen Arbeitsräumen}
      \CVItemListEnd
  \CVSubHeadingListEnd

%-----ZERTIFIZIERUNGEN---------------------------------------------------------
\section{Zertifizierungen}
  \CVSubHeadingListStart
      \CVSubheading
      {International Conference on Advancements in Automation, Robotics, and Sensing}{Dez. 2018}
      {PSG College of Technology}{}
    \CVSubheading
      {Create Your First Chatbot with Rasa and Python}{Juli 2020}
      {Coursera - Credential ID: ZY7KXX6ED6UH}{}
    \CVSubheading
      {AI For Everyone}{Juni 2020}
      {Coursera - Credential ID: DXK5S8JHHS3Q}{}
    \CVSubheading
      {Aerial Surveying and Mapping using Drone Technology}{Juni 2020}
      {Aviocian Technologies Pvt. Ltd.}{}
    \CVSubheading
      {Machine Learning for All}{Juni 2020}
      {Coursera - Credential ID: 7AVKB8YE6Q4W}{}
    \CVSubheading
      {Techno Workshop Series on Machine Learning}{Feb. 2019}
      {Indian Institute of Technology, Madras}{}

  \CVSubHeadingListEnd
  %-----HOBBYS UND INTERESSEN----------------------------------------------------
\section{Hobbys und Interessen}
 \begin{itemize}[leftmargin=0.5cm, label={}]
    \small{\item{
     \textbf{Robotik \& Technik}{: Open-Source-Projekte, ROS-Community, 3D-Druck, Prototyping} \\
     \textbf{Weiterbildung}{: KI-Konferenzen, Webinare zu autonomen Systemen, technische Literatur} \\
     \textbf{Kreativ \& Aktiv}{: Fotografie, Reisen, kultureller Austausch} \\
    }}
 \end{itemize}

%------------------------------------------------------------------------------
\end{document}
